% ------------------------------------------------------------------------
% TEMPLATE OFICIAL - REVISTA GESTADI (PPG LETRAS/UFT)
% Correção: Títulos em Times New Roman (rmfamily) e Tamanho 12
% ------------------------------------------------------------------------

\documentclass[
	article,			% Artigo acadêmico
	12pt,				% Tamanho da fonte do corpo do texto
	oneside,			% Impressão apenas frente
	a4paper,			% Tamanho do papel
	english,			% Idioma adicional
	brazil				% Idioma principal
	]{abntex2}

% ---
% PACOTES FUNDAMENTAIS
% ---
\usepackage{cmap}				% Mapeamento de caracteres
\usepackage{lmodern}			% Fonte base
\usepackage{mathptmx}           % FONTE TIMES NEW ROMAN (Obrigatória)
\usepackage[T1]{fontenc}		% Seleção de códigos de fonte
\usepackage[utf8]{inputenc}		% Codificação do arquivo
\usepackage{indentfirst}		% Indenta o primeiro parágrafo
\usepackage{color}				% Controle de cores
\usepackage{graphicx}			% Inclusão de imagens
\usepackage{microtype} 			% Melhorias de justificação
\usepackage{lipsum}				% Gerador de texto de exemplo
\usepackage[alf]{abntex2cite}	% Citações ABNT (Autor, Data)

% ---
% CONFIGURAÇÃO DE MARGENS (Norma: Esq/Dir 3.0, Sup/Inf 2.5)
% ---
\usepackage{geometry}
\geometry{
    left=3.0cm,
    right=3.0cm,
    top=2.5cm,
    bottom=2.5cm
}

% ---
% FORÇAR TÍTULOS EM TIMES NEW ROMAN E TAMANHO 12
% ---
% O padrão do LaTeX é Sans Serif e tamanhos grandes. 
% Os comandos abaixo forçam a fonte RM (Roman/Times), Negrito e Tamanho 12.

\renewcommand{\ABNTEXchapterfont}{\rmfamily\bfseries}
\renewcommand{\ABNTEXsectionfont}{\rmfamily\bfseries\fontsize{12}{15}\selectfont}
\renewcommand{\ABNTEXsubsectionfont}{\rmfamily\bfseries\fontsize{12}{15}\selectfont}
\renewcommand{\ABNTEXsubsubsectionfont}{\rmfamily\bfseries\fontsize{12}{15}\selectfont}

% ---
% CONFIGURAÇÃO DE CABEÇALHO E RODAPÉ (PERSONALIZADO)
% ---
\makepagestyle{gestadi}

% Cabeçalho: Apenas número da página à direita (Topo)
\makeevenhead{gestadi}{}{}{\thepage}
\makeoddhead{gestadi}{}{}{\thepage}

% Rodapé: Texto centralizado, Times New Roman 10, Simples
\newcommand{\textorodape}{%
    \fontsize{10}{12}\selectfont % Fonte 10, entrelinha simples
    \centering 
    Gestadi - Revista do Grupo de Estudo de Análise do Discurso \\ 
    Volume 1, Número 3, 2025 2º semestre (Projeções discursivas)
}
\makeevenfoot{gestadi}{}{\textorodape}{}
\makeoddfoot{gestadi}{}{\textorodape}{}

% ---
% AJUSTES ESPECÍFICOS DA NORMA
% ---
% 1. Citação Longa: Recuo de 3 cm (Norma pede 3cm)
\setlength{\ABNTEXcitacaorecuo}{3cm}

% 2. Citação Longa: Fonte tamanho 10
\renewcommand{\ABNTEXfontereduzida}{\fontsize{10}{12}\selectfont} 

% 3. Espaçamento entre linhas 1,5 no texto
\OnehalfSpacing 

% 4. Recuo de parágrafo 1,25 cm
\setlength{\parindent}{1.25cm}

% ---
% INÍCIO DO DOCUMENTO
% ---
\begin{document}

% Ativa o estilo de cabeçalho e rodapé personalizado
\pagestyle{gestadi}

% ---
% TÍTULOS E AUTORES
% ---
\begin{center}
    % TÍTULO EM PORTUGUÊS (Caixa alta, Negrito, 12, Esp 1.5)
    {\fontsize{12}{18}\selectfont \textbf{\MakeUppercase{Título do Artigo em Português}}}
    
    \vspace{18pt} 
    
    % TITLE IN ENGLISH (Caixa alta, Negrito, 12, Esp 1.5)
    {\fontsize{12}{18}\selectfont \textbf{\MakeUppercase{Title of the Article in English}}}
\end{center}

\vspace{1cm}

% AUTORES
\begin{center}
    Nome do Autor 1\footnote{Titulação, Instituição. Email: autor1@email.com} \\
    Instituição do Autor 1
    
    \vspace{0.5cm}
    
    Nome do Autor 2\footnote{Titulação, Instituição. Email: autor2@email.com} \\
    Instituição do Autor 2
\end{center}

\vspace{1cm}

% ---
% RESUMO E ABSTRACT
% ---
\begin{SingleSpace} 
    \fontsize{11}{11}\selectfont % Tamanho 11
    \noindent 
    \textbf{Resumo:} \lipsum[1] (Texto em Times New Roman 11, simples, sem recuo).
    
    \vspace{0.3cm}
    \noindent
    \textbf{Palavras-chave:} Palavra 1; Palavra 2; Palavra 3.
    
    \vspace{0.8cm} 
    
    \noindent
    \textbf{Abstract:} \lipsum[2] (Text in Times New Roman 11, single spacing, no indent).
    
    \vspace{0.3cm}
    \noindent
    \textbf{Keywords:} Word 1; Word 2; Word 3.
\end{SingleSpace}

\vspace{1cm}

% DATAS
\begin{flushright}
    \textbf{Submetido em dia de mês de 2020.}\\
    \textbf{Aprovado em dia de mês de 2020.}
\end{flushright}

\vspace{1cm}

% ---
% CORPO DO TEXTO
% ---

% Introdução não numerada
\section*{Introdução}
\addcontentsline{toc}{section}{Introdução}

A introdução não deve ser numerada. Observe agora que o título "Introdução" acima está em Times New Roman, Negrito e do mesmo tamanho do texto (12), conforme solicitado.

% Seção numerada
\section{Desenvolvimento do Artigo}

Este título "Desenvolvimento do Artigo" também deve estar em Times New Roman. O LaTeX padrão usaria uma fonte sem serifa e muito maior, mas os comandos \texttt{renewcommand} no início do código corrigiram isso.

Exemplo de citação longa (Recuo 3cm, Fonte 10):
\begin{citacao}
Esta é uma citação longa formatada corretamente com Times New Roman, tamanho 10 e recuo de 3cm da margem esquerda. \cite[p. 10]{Eagleton2003}
\end{citacao}

\section{Resultados}

\begin{table}[htb]
    \centering
    \caption{Exemplo de tabela (Times 10)}
    \label{tab:exemplo}
    {\fontsize{10}{12}\selectfont
    \begin{tabular}{lcccc}
        \hline
        \textbf{ÁREAS} & \textbf{2018} & \textbf{2019} & \textbf{TOTAL} \\
        \hline
        Região 1 & 7 & 8 & 15 \\
        Região 2 & 5 & 5 & 10 \\
        \hline
    \end{tabular}
    }
    \par \vspace{0.2cm}
    {\fontsize{10}{12}\selectfont \textbf{Fonte:} \citeonline{Peterson2015}.}
\end{table}

% Conclusão não numerada
\section*{Considerações Finais}
\addcontentsline{toc}{section}{Considerações Finais}

Texto das considerações finais.

% ---
% REFERÊNCIAS
% ---
\vspace{1cm}
\section*{Referências}
\addcontentsline{toc}{section}{Referências}

\begingroup
    \SingleSpacing
    \bibliography{referencias}
\endgroup

\vspace{1cm}
{\fontsize{10}{12}\selectfont
\section*{Apêndices e Anexos}
Texto dos apêndices em fonte 10 e espaçamento simples.
}

\end{document}