% ------------------------------------------------------------------------
% TEMPLATE OFICIAL - REVISTA GESTADI (PPG LETRAS/UFT)
% Baseado nas normas fornecidas (Google Docs/Texto)
% ------------------------------------------------------------------------

\documentclass[
	article,			% Artigo acadêmico
	12pt,				% Tamanho da fonte do corpo do texto
	oneside,			% Impressão apenas frente
	a4paper,			% Tamanho do papel
	english,			% Idioma adicional
	brazil				% Idioma principal
]{abntex2}

% ---
% PACOTES FUNDAMENTAIS
% ---
\usepackage{cmap}				% Mapeamento de caracteres
\usepackage{lmodern}			% Fonte base
\usepackage{mathptmx}           % FONTE TIMES NEW ROMAN (Obrigatória)
\usepackage[T1]{fontenc}		% Seleção de códigos de fonte
\usepackage[utf8]{inputenc}		% Codificação do arquivo
\usepackage{indentfirst}		% Indenta o primeiro parágrafo
\usepackage{color}				% Controle de cores
\usepackage{graphicx}			% Inclusão de imagens
\usepackage{microtype} 			% Melhorias de justificação
\usepackage{lipsum}				% Gerador de texto de exemplo
\usepackage[alf]{abntex2cite}	% Citações ABNT (Autor, Data)

% ---
% CONFIGURAÇÃO DE MARGENS (Conforme normas)
% Sup/Inf: 2,5cm | Esq/Dir: 3,0cm
% ---
\usepackage{geometry}
\geometry{
	left=3.0cm,
	right=3.0cm,
	top=2.5cm,
	bottom=2.5cm
}

% ---
% FORÇAR TÍTULOS EM TIMES NEW ROMAN E TAMANHO 12
% ---
% O padrão do LaTeX é Sans Serif e tamanhos grandes. 
% Os comandos abaixo forçam a fonte RM (Roman/Times), Negrito e Tamanho 12.

\renewcommand{\ABNTEXchapterfont}{\rmfamily\bfseries}
\renewcommand{\ABNTEXsectionfont}{\rmfamily\bfseries\fontsize{12}{15}\selectfont}
\renewcommand{\ABNTEXsubsectionfont}{\rmfamily\bfseries\fontsize{12}{15}\selectfont}
\renewcommand{\ABNTEXsubsubsectionfont}{\rmfamily\bfseries\fontsize{12}{15}\selectfont}

% ---
% CONFIGURAÇÃO DE CABEÇALHO E RODAPÉ (PERSONALIZADO)
% ---
\makepagestyle{gestadi}

% Cabeçalho: Apenas número da página à direita (Topo)
\makeevenhead{gestadi}{}{}{\thepage}
\makeoddhead{gestadi}{}{}{\thepage}

% Rodapé: Texto centralizado, Times New Roman 10, Simples
\newcommand{\textorodape}{%
	\fontsize{10}{12}\selectfont % Fonte 10, entrelinha simples
	\centering 
	Gestadi - Revista do Grupo de Estudo de Análise do Discurso \\ 
	Volume 1, Número 3, 2025 2º semestre (Projeções discursivas)
}
\makeevenfoot{gestadi}{}{\textorodape}{}
\makeoddfoot{gestadi}{}{\textorodape}{}

% ---
% AJUSTES ESPECÍFICOS DA NORMA (Onde difere da ABNT padrão)
% ---

% 1. Citação Longa: Recuo de 3 cm (Norma pede 3cm, ABNT padrão é 4cm)
\setlength{\ABNTEXcitacaorecuo}{3cm}

% 2. Citação Longa: Fonte tamanho 10
\renewcommand{\ABNTEXfontereduzida}{\fontsize{10}{12}\selectfont} 

% 3. Espaçamento entre linhas 1,5 no texto
\OnehalfSpacing 

% 4. Recuo de parágrafo 1,25 cm
\setlength{\parindent}{1.25cm}

% ---
% INÍCIO DO DOCUMENTO
% ---
\begin{document}
	
	% Ativa o estilo de cabeçalho e rodapé personalizado
	\pagestyle{gestadi}
	
	% ---
	% TÍTULOS E AUTORES (Manual para garantir formatação exata)
	% ---
	\begin{center}
		% TÍTULO EM PORTUGUÊS (Caixa alta, Negrito, 12, Esp 1.5)
		{\fontsize{12}{18}\selectfont \textbf{\MakeUppercase{Título do Artigo em Português: Subtítulo se Houver}}}
		
		\vspace{18pt} % Uma linha de espaço (considerando entrelinha 1.5 de 12pt = 18pt)
		
		% TITLE IN ENGLISH (Caixa alta, Negrito, 12, Esp 1.5)
		{\fontsize{12}{18}\selectfont \textbf{\MakeUppercase{Title of the Article in English}}}
	\end{center}
	
	\vspace{1cm}
	
	% AUTORES (Alinhamento não especificado explicitamente como direita ou centro no texto, 
	% mas visualmente em artigos costuma ser direita ou centro. Texto diz: "Duas linhas após identificação")
	%\begin{center}
	%	Nome do Autor 1\footnote{Titulação, Instituição. Email: autor1@email.com} \\
	%	Instituição do Autor 1
	%	
	%	\vspace{0.5cm}
	%	
	%	Nome do Autor 2\footnote{Titulação, Instituição. Email: autor2@email.com} \\
	%	Instituição do Autor 2
	%\end{center}

	% Autores à direita
	\begin{flushright}
		Wilda Barbosa Nóia\footnote{Mestranda em Letras. Universidade Federal do Tocantins. Email: exemplo@uft.edu.br} \\
		Universidade Federal do Tocantins
		\vspace{0.5cm}
		
		Nome do Orientador\footnote{Professor Doutor. Universidade Federal do Tocantins.} \\
		Universidade Federal do Tocantins
	\end{flushright}
	
	\vspace{1cm} % "Duas linhas após a identificação do autor" (aprox 1cm visual)
	
	% ---
	% RESUMO (Fonte 11, Simples, Sem adentramento)
	% ---
	\begin{SingleSpace} 
		\fontsize{11}{11}\selectfont % Tamanho 11
		\noindent % Sem adentramento
		\textbf{Resumo:} \lipsum[1] (O resumo deve ter entre 150 e 250 palavras, fonte Times New Roman 11, espaçamento simples, sem recuo na primeira linha).
		
		\vspace{0.3cm}
		\noindent
		\textbf{Palavras-chave:} Palavra 1; Palavra 2; Palavra 3 (3 a 5, separadas por ponto-e-vírgula).
		
		\vspace{0.8cm} % Uma linha após as palavras-chave
		
		\noindent
		\textbf{Abstract:} \lipsum[2] (Translation of the abstract. Must follow the same formatting rules: 11pt, single spacing, no indentation).
		
		\vspace{0.3cm}
		\noindent
		\textbf{Keywords:} Word 1; Word 2; Word 3 (3 to 5, separated by colon).
	\end{SingleSpace}
	
	\vspace{1cm}
	
	% DATAS (Inseridas pelos editores, mas presentes no template)
	\begin{flushright}
		\textbf{Submetido em dia de mês de 2020.}\\
		\textbf{Aprovado em dia de mês de 2020.}
	\end{flushright}
	
	\vspace{1cm}
	
	% ---
	% CORPO DO TEXTO
	% ---
	
	% INTRODUÇÃO (Não numerada, Negrito, Esquerda)
	\section*{Introdução}
	\addcontentsline{toc}{section}{Introdução} % Adiciona aos bookmarks do PDF
	
	A introdução não deve ser numerada. O texto deve ser escrito em fonte Times New Roman, tamanho 12, espaçamento entre linhas 1,5, margens superior/inferior 2,5cm e laterais 3,0cm.
	
	% SEÇÕES NUMERADAS
	\section{Desenvolvimento do Artigo}
	
	Os títulos das seções devem ser numerados com algarismos arábicos. Citações no corpo do texto com até três linhas: ``Exemplo de citação curta'' \cite[p. 12]{Eagleton2003}.
	
	Citações com mais de três linhas devem seguir a regra específica da revista (Recuo de 3cm):
	
	\begin{citacao}
		Esta é uma citação longa com mais de três linhas. Conforme as normas especificadas, ela deve vir em bloco, espaçamento simples, fonte Times New Roman tamanho 10, com recuo de 3 cm da margem esquerda do texto. Note que o padrão ABNT comum seria 4 cm, mas este template está ajustado para 3 cm. \cite[p. 45]{Resende2004}
	\end{citacao}
	
	\section{Resultados e Discussão}
	
	Exemplo de inserção de tabela. As tabelas devem ter fonte tamanho 10.
	
	\begin{table}[htb]
		\centering
		\caption{Exemplo de tabela.}
		\label{tab:exemplo}
		% Ajuste de fonte para tamanho 10 dentro da tabela
		{\fontsize{10}{12}\selectfont
			\begin{tabular}{lcccc}
				\hline
				\textbf{ÁREAS} & \textbf{2017} & \textbf{2018} & \textbf{2019} & \textbf{TOTAL} \\
				\hline
				Região 1 & 2 & 7 & 8 & 17 \\
				Região 2 & 5 & 8 & 5 & 18 \\
				Região 3 & 3 & 10 & 6 & 19 \\
				\hline
				\textbf{TOTAL} & \textbf{10} & \textbf{25} & \textbf{19} & \textbf{54} \\
				\hline
			\end{tabular}
		}
		\par \vspace{0.2cm}
		{\fontsize{10}{12}\selectfont \textbf{Fonte:} \citeonline{Peterson2015}.}
	\end{table}
	
	\subsection{Exemplo de Imagem}
	
	Figuras devem ser centralizadas com legenda numerada.
	
	\begin{figure}[htb]
		\centering
		\caption{Mapa da Região Metropolitana}
		% Insere imagem
		\includegraphics[width=0.8\textwidth]{Imagens/imagem-01.jpg}
		%\fbox{\begin{minipage}[c][3cm]{8cm} \centering (Espaço para Imagem) \Send{minipage}}
		\par \vspace{0.2cm}
		{\fontsize{10}{12}\selectfont \textbf{Fonte:} https://secom.to.gov.br/}
	\end{figure}
	
	% CONCLUSÃO (Não numerada)
	\section*{Considerações Finais}
	\addcontentsline{toc}{section}{Considerações Finais}
	
	A seção de considerações finais / conclusão não deve ser numerada. Fonte 12, espaçamento 1,5. Deve ser sucinta e responder aos objetivos da introdução.
	
	% ---
	% REFERÊNCIAS
	% ---
	\vspace{1cm}
	\section*{Referências}
	\addcontentsline{toc}{section}{Referências}
	
	% Configuração das referências: Espaçamento simples, alinhado à esquerda
	\begingroup
	\SingleSpacing
	\bibliography{referencias}
	\endgroup
	
	% ---
	% APÊNDICES E ANEXOS (Se houver)
	% ---
	\vspace{1cm}
	% Fonte 10, espaçamento simples para apêndices conforme norma
	{\fontsize{10}{12}\selectfont
		\section*{Apêndices e Anexos}
		Apêndices e anexos devem vir após as referências, em fonte Times New Roman, tamanho 10, espaçamento simples.
	}
	
\end{document}