%% abtex2-modelo-trabalho-academico.tex, v-1.9.7 laurocesar
%% Customizado para UFT - PPG Letras Porto Nacional

\documentclass[
	% -- opções da classe memoir --
	12pt,				% tamanho da fonte
	openright,			% capítulos começam em pág ímpar (insere página vazia se preciso)
	twoside,			% para impressão em verso e anverso. Oposto a oneside
	a4paper,			% tamanho do papel. 
	% -- opções da classe abntex2 --
	chapter=TITLE,		% títulos de capítulos convertidos em letras maiúsculas
	section=TITLE,		% títulos de seções convertidos em letras maiúsculas
	% -- opções do pacote babel --
	english,			% idioma adicional para hifenização
	french,				% idioma adicional para hifenização
	spanish,			% idioma adicional para hifenização
	brazil				% o último idioma é o principal do documento
	]{abntex2}

% ---
% Pacotes básicos 
% ---
\usepackage{lmodern}			% Usa a fonte Latin Modern			
\usepackage[T1]{fontenc}		% Selecao de codigos de fonte.
\usepackage[utf8]{inputenc}		% Codificacao do documento (conversion to UTF8)
\usepackage{indentfirst}		% Indenta o primeiro parágrafo de cada seção.
\usepackage{color}				% Controle das cores
\usepackage{graphicx}			% Inclusão de gráficos
\usepackage{microtype} 			% para melhorias de justificação
\usepackage{lipsum}				% para geração de dummy text

% ---
% Pacotes de citações
% ---
\usepackage[brazilian,hyperpageref]{backref}	 % Paginas com as citações na bibl
\usepackage[alf]{abntex2cite}	% Citações padrão ABNT

% --- 
% CONFIGURAÇÕES DE PACOTES
% --- 

% Configurações do pacote backref
% Usado sem a opção hyperpageref de backref
\renewcommand{\backrefpagesname}{Citado na(s) página(s):~}
% Texto padrão antes do número das páginas
\renewcommand{\backref}{}
% Define os textos da citação
\renewcommand*{\backrefalt}[4]{
	\ifcase #1 %
		Nenhuma citação no texto.%
	\or
		Citado na página #2.%
	\else
		Citado #1 vezes nas páginas #2.%
	\fi}%
% ---

% ---
% Informações de dados para CAPA e FOLHA DE ROSTO
% ---
\titulo{TÍTULO DA DISSERTAÇÃO: subtítulo se houver (Opcional)}
\autor{NOME COMPLETO DO AUTOR}
\local{PORTO NACIONAL -- TO}
\data{2025}
\orientador{Prof. Dr. Nome do Orientador}
\coorientador{Prof. Dr. Nome do Coorientador (se houver)}
\instituicao{%
  UNIVERSIDADE FEDERAL DO TOCANTINS -- UFT
  \par
  CÂMPUS DE PORTO NACIONAL
  \par
  PROGRAMA DE PÓS-GRADUAÇÃO EM LETRAS}
\tipotrabalho{Dissertação (Mestrado)}
% O preambulo deve conter o tipo do trabalho, o objetivo, o nome da instituição e a área de concentração 
\preambulo{Dissertação apresentada ao Programa de Pós-Graduação em Letras da Universidade Federal do Tocantins (Câmpus Porto Nacional), como requisito parcial para obtenção do título de Mestre em Letras. \newline\newline Área de Concentração: Estudos Literários / Estudos Linguísticos.}
% ---

% ---
% Configurações de aparência do PDF final
% ---

% alterando o aspecto da cor azul
\definecolor{blue}{RGB}{41,5,195}

% informações do PDF
\makeatletter
\hypersetup{
     	%pagebackref=true,
		pdftitle={\@title}, 
		pdfauthor={\@author},
    	pdfsubject={\imprimirpreambulo},
	    pdfcreator={LaTeX with abnTeX2},
		pdfkeywords={abnt}{latex}{abntex2}{trabalho acadêmico}{uft}, 
		colorlinks=true,       		% false: box links; true: colored links
    	linkcolor=black,          	% color of internal links
    	citecolor=black,        		% color of links to bibliography
    	filecolor=magenta,      		% color of file links
		urlcolor=black,
		bookmarksdepth=4
}
\makeatother
% --- 

% ---
% Espaçamentos entre linhas e parágrafos 
% ---

% O tamanho do parágrafo é dado por:
\setlength{\parindent}{1.25cm} % Padrão ABNT/UFT

% Controle do espaçamento entre um parágrafo e outro:
\setlength{\parskip}{0.2cm}  % tente também \onelineskip

% ---
% Compila o índice
% ---
\makeindex
% ---

% ----
% INÍCIO DO DOCUMENTO
% ----
\begin{document}

% Seleciona o idioma principal (para hifenização e nomes de documentos)
\selectlanguage{brazil}

% Retira espaço extra obsoleto entre as frases.
\frenchspacing 

% ----------------------------------------------------------
% ELEMENTOS PRÉ-TEXTUAIS
% ----------------------------------------------------------
% \pretextual

% ---
% Capa
% ---
\imprimircapa
% ---

% ---
% Folha de rosto
% (o * indica que haverá a ficha bibliográfica)
% ---
\imprimirfolhaderosto*
% ---

% ---
% Inserir a ficha bibliografica
% ---
% A ficha catalográfica oficial deve ser gerada pelo sistema da biblioteca da UFT (SISBIB).
% Este é apenas um placeholder para o layout.
\begin{fichacatalografica}
	\sffamily
	\vspace*{\fill}					% Posição vertical
	\begin{center}					% Minipage Centralizado
	\fbox{\begin{minipage}[c][8cm]{13.5cm}		% Largura
	\small
	\imprimirautor
	%Sobrenome, Nome do autor
	
	\hspace{0.5cm} \imprimirtitulo  / \imprimirautor. --
	\imprimirlocal, \imprimirdata-
	
	\hspace{0.5cm} \pageref{LastPage} p. : il. (algumas color.) ; 30 cm.
	
	\hspace{0.5cm} \imprimirorientadorRotulo~\imprimirorientador\\
	
	\hspace{0.5cm}
	\parbox[t]{\textwidth}{\imprimirtipotrabalho~--~\imprimirinstituicao,
	\imprimirdata.}
	
	\hspace{0.5cm}
		1. Palavra-chave1.
		2. Palavra-chave2.
		2. Palavra-chave3.
		I. Orientador.
		II. Universidade Federal do Tocantins.
		III. Título.
	\end{minipage}}
	\end{center}
\end{fichacatalografica}
% ---

% ---
% Folha de aprovação
% ---
\begin{folhadeaprovacao}

  \begin{center}
    {\ABNTEXchapterfont\large\imprimirautor}

    \vspace*{\fill}\center
    \begin{center}
      \ABNTEXchapterfont\bfseries\Large\imprimirtitulo
    \end{center}
    \vspace*{\fill}
    
    \hspace{.45\textwidth}
    \begin{minipage}{.5\textwidth}
        \imprimirpreambulo
    \end{minipage}%
    \vspace*{\fill}
   \end{center}
        
   \begin{center}
       Aprovada em: \_\_ de \_\_\_\_\_\_ de \_\_\_\_ .
       \vspace*{1cm}
   
       \textbf{BANCA EXAMINADORA}
       \vspace*{1cm}
       
       \assinatura{\textbf{\imprimirorientador} \\ Orientador - UFT} 
       \assinatura{\textbf{Prof. Dr. Membro Interno} \\ UFT}
       \assinatura{\textbf{Prof. Dr. Membro Externo} \\ Instituição de Origem}
       %\assinatura{\textbf{Professor} \\ Convidado 4}
      
   \end{center}
  
\end{folhadeaprovacao}
% ---

% ---
% Dedicatória
% ---
\begin{dedicatoria}
   \vspace*{\fill}
   \centering
   \noindent
   \textit{Dedico este trabalho a todos que...} \vspace*{1.5cm}
\end{dedicatoria}
% ---

% ---
% Agradecimentos
% ---
\begin{agradecimentos}
Agradeço primeiramente à UFT...

Ao meu orientador...

À CAPES/CNPq (se for bolsista)...
\end{agradecimentos}
% ---

% ---
% Epígrafe
% ---
\begin{epigrafe}
    \vspace*{\fill}
	\begin{flushright}
		\textit{``A língua é um sistema de signos que exprimem ideias...''\\
		(Ferdinand de Saussure)}
	\end{flushright}
\end{epigrafe}
% ---

% ---
% RESUMOS
% ---

% resumo em português
\setlength{\absparsep}{18pt} % ajusta o espaçamento dos parágrafos do resumo
\begin{resumo}
 O resumo deve ressaltar o objetivo, o método, os resultados e as conclusões do documento. Deve ser composto de uma sequência de frases concisas e objetivas (e não de uma simples enumeração de tópicos). Deve ter entre 150 e 500 palavras (Norma UFT/ABNT NBR 6028).
 
 \textbf{Palavras-chave}: Linguística. Literatura. Ensino. Tocantins.
\end{resumo}

% resumo em inglês
\begin{resumo}[Abstract]
 \begin{otherlanguage*}{english}
   This is the english abstract. It must follow the same rules as the portuguese abstract.
   
   \textbf{Keywords}: Linguistics. Literature. Teaching. Tocantins.
 \end{otherlanguage*}
\end{resumo}

% ---
% Listas de ilustrações e tabelas
% ---
\pdfbookmark[0]{\listfigurename}{lof}
\listoffigures*
\cleardoublepage

\pdfbookmark[0]{\listtablename}{lot}
\listoftables*
\cleardoublepage

% ---
% Sumário
% ---
\pdfbookmark[0]{\contentsname}{toc}
\tableofcontents*
\cleardoublepage
% ---

% ----------------------------------------------------------
% ELEMENTOS TEXTUAIS
% ----------------------------------------------------------
\textual

% ----------------------------------------------------------
% Introdução (exemplo de capítulo sem numeração, mas presente no sumário)
% ----------------------------------------------------------
\chapter[Introdução]{INTRODUÇÃO}
%\addcontentsline{toc}{chapter}{INTRODUÇÃO} % Se usar chapter*, descomente esta linha
A introdução deve situar o leitor no contexto do tema pesquisado.

\lipsum[1-3]

% ----------------------------------------------------------
% Capitulo 1
% ----------------------------------------------------------
\chapter{FUNDAMENTAÇÃO TEÓRICA}
Neste capítulo, aborda-se a teoria de Saussure e seus desdobramentos.

\section{A Dicotomia Língua e Fala}
Conforme a ABNT NBR 10520 e Manual UFT:

Citação direta curta (até 3 linhas): ``O signo linguístico une não uma coisa e uma palavra, mas um conceito e uma imagem acústica'' \cite[p. 66]{saussure1916}.

Citação direta longa (mais de 3 linhas), deve ter recuo de 4cm, tamanho 10 e espaçamento simples:

\begin{citacao}
A linguística tem por único e verdadeiro objeto a língua considerada em si mesma e por si mesma. Esta afirmação final do Curso de Linguística Geral marcou o início do estruturalismo. \cite[p. 100]{saussure1916}
\end{citacao}

\lipsum[4-6]

% ----------------------------------------------------------
% Capitulo 2
% ----------------------------------------------------------
\chapter{METODOLOGIA}
\lipsum[7-8]

% ----------------------------------------------------------
% Capitulo 3
% ----------------------------------------------------------
\chapter{ANÁLISE DOS RESULTADOS}
\lipsum[9-12]

% ----------------------------------------------------------
% Conclusão
% ----------------------------------------------------------
\chapter{CONSIDERAÇÕES FINAIS}
%\addcontentsline{toc}{chapter}{CONSIDERAÇÕES FINAIS}
\lipsum[13]

% ----------------------------------------------------------
% ELEMENTOS PÓS-TEXTUAIS
% ----------------------------------------------------------
\postextual
% ----------------------------------------------------------

% ----------------------------------------------------------
% Referências bibliográficas
% ----------------------------------------------------------
\bibliography{referencias}

% ----------------------------------------------------------
% Apêndices (opcional)
% ----------------------------------------------------------
\begin{apendicesenv}
\partapendices
\chapter{Questionário aplicado}
\lipsum[14]
\end{apendicesenv}

% ----------------------------------------------------------
% Anexos (opcional)
% ----------------------------------------------------------
\begin{anexosenv}
\partanexos
\chapter{Documentos da Escola}
\lipsum[15]
\end{anexosenv}

\end{document}