\documentclass[12pt,openright,twoside,a4paper,english,french,spanish,brazil]{abntex2}

\usepackage[utf8]{inputenc}
\usepackage[T1]{fontenc}
\usepackage{cmap}
\usepackage{lmodern}
\usepackage{indentfirst}
\usepackage{graphicx}
\usepackage{color}
\usepackage{microtype}
\usepackage{lipsum}

% --- Informações de dados para CAPA e FOLHA DE ROSTO ---
\titulo{TÍTULO DA DISSERTAÇÃO: subtítulo se houver}
\autor{Seu Nome Completo}
\local{Porto Nacional - TO}
\data{2025}
\orientador{Prof. Dr. Nome do Orientador}
\instituicao{%
  Universidade Federal do Tocantins -- UFT
  \par
  Câmpus de Porto Nacional
  \par
  Programa de Pós-Graduação em Letras}
\tipotrabalho{Dissertação (Mestrado)}
\preambulo{Dissertação apresentada ao Programa de Pós-Graduação em Letras da Universidade Federal do Tocantins, como requisito parcial para obtenção do título de Mestre em Letras.}

% --- Início do Documento ---
\begin{document}
\frenchspacing

% 1. Elementos Pré-Textuais
\imprimircapa
\imprimirfolhaderosto*

% Ficha Catalográfica (Exemplo)
\begin{fichacatalografica}
    \vspace*{\fill}
    \begin{center}
    \fbox{\begin{minipage}[c][8cm]{13.5cm}
    \small
    Sobrenome, Nome.
    \hspace{0.5cm} \imprimirtitulo / \imprimirautor. -- \imprimirlocal, \imprimirdata.
    \hspace{0.5cm} \pageref{LastPage} p. : il. (algumas color.) ; 30 cm.\\
    \hspace{0.5cm} \imprimirorientadorRotulo~\imprimirorientador\\
    \hspace{0.5cm} \imprimirtipotrabalho~--~\imprimirinstituicao, \imprimirdata.\\
    \hspace{0.5cm} 1. Palavra-chave1. 2. Palavra-chave2. I. Orientador. II. Universidade Federal do Tocantins. III. Título.
    \end{minipage}}
    \end{center}
\end{fichacatalografica}

% Folha de Aprovação
\begin{folhadeaprovacao}
  \begin{center}
    {\ABNTEXchapterfont\large\imprimirautor}
    \vspace*{\fill}\par
    {\ABNTEXchapterfont\bfseries\Large\imprimirtitulo}
    \vspace*{\fill}
    \par
    \begin{minipage}{.5\textwidth}
        \imprimirpreambulo
    \end{minipage}%
    \vspace*{\fill}
   \end{center}
   \begin{center}
    Trabalho aprovado. \imprimirlocal, \today:
   \end{center}
   \assinatura{\textbf{\imprimirorientador} \\ Orientador} 
   \assinatura{\textbf{Prof. Dr. Membro da Banca 1} \\ UFT}
   \assinatura{\textbf{Prof. Dr. Membro da Banca 2} \\ Convidado Externo}
\end{folhadeaprovacao}

\begin{dedicatoria}
   \vspace*{\fill}
   \centering
   \noindent
   \textit{Dedico este trabalho aos meus pais e professores...}
   \vspace*{2cm}
\end{dedicatoria}

\begin{agradecimentos}
Agradeço à UFT, ao Câmpus de Porto Nacional e ao meu orientador...
\end{agradecimentos}

\begin{resumo}
Segundo as normas da UFT, o resumo deve conter objetivo, método, resultados e conclusões do documento.
\vspace{\onelineskip}
\noindent
\textbf{Palavras-chave}: Letras. Literatura/Linguística. Tocantins.
\end{resumo}

\begin{resumo}[Abstract]
 \begin{otherlanguage*}{english}
   This is the abstract in English.
   \vspace{\onelineskip}
   \noindent
   \textbf{Keywords}: Letters. Literature/Linguistics. Tocantins.
 \end{otherlanguage*}
\end{resumo}

\pdfbookmark[0]{\listfigurename}{lof}
\listoffigures*
\cleardoublepage

\pdfbookmark[0]{\contentsname}{toc}
\tableofcontents*
\cleardoublepage

% 2. Elementos Textuais
\textual

\chapter{Introdução}
A introdução deve apresentar o tema, a justificativa e os objetivos (geral e específicos).
\lipsum[1-3]

\chapter{Fundamentação Teórica}
\lipsum[4-8]

\chapter{Metodologia}
Descrição detalhada dos procedimentos metodológicos.
\lipsum[9-10]

\chapter{Análise dos Resultados}
\lipsum[11-15]

\chapter{Considerações Finais}
\lipsum[16-17]

% 3. Elementos Pós-Textuais
\postextual
\bibliography{referencias}

\end{document}